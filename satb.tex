\documentclass{scrartcl}

\usepackage{amsmath, amssymb}

\newtheorem{example}{Example}

\title{SATB}
\subtitle{A Colourful Game of Musical Puzzles}
\author{Michael Purcell}
\date{}

\begin{document}
\maketitle
\section{Introduction}\label{section:introduction}
SATB is a game for 1-5 players who work together to create a \emph{composition}.
A composition is a two-dimensional array of \emph{colours}.   
Each row of a composition is called a \emph{voice} while
each column is called a \emph{beat}.
Players create compositions by choosing which colour each voice will play on
each beat. 

\section{Colours and Rests}
There are five colours in an SATB composition:
red ($R$), yellow ($Y$), green ($G$), blue ($B$), and white ($W$).
We impose a structure on these five colours. This structure is depicted on the game
board by solid and dashed lines. 
If two colours are connected by a solid line then we say that they are \emph{adjacent}.
If two colours connected by a dashed line then they are not adjacent.
Given a composition $\mathbf{C} = [C_{ij}]$, we say that $i$th voice
\emph{plays a colour} on the $j$th beat if $C_{ij} \in \{R, Y, G, B, W\}$.

There is a sixth space on the game board ($X$) which represents a rest.
Given a composition $\mathbf{C} = [C_{ij}]$,
we say that the $i$th voice \emph{rests} on the $j$th beat if $C_{ij} = X$.

\section{Composition Rules}
Each composition must satisfy a set of rules which ensure that all of the voices are
both individually interesting and mutually consistent with one another.
 
\subsection{Rhythm}
As described in Section \ref{section:introduction}, each composition is divided up into a sequence of beats. On each beat, each voice must either play a colour or rest.
Furthermore, the voices must collectively obey the \emph{continuity} rules:
\begin{description}
	\item[R1] No more than one voice may rest on each beat.
	\item[R2] No voice may rest on more than one consecutive beat.
\end{description}

\subsection{Melody}
A \emph{melody} is a sequence of colours and rests played by a single voice.  A \emph{repeat} is a when a voice plays the same colour on two consecutive beats. A \emph{step} is when a voice plays two adjacent colours on two consecutive beats. A \emph{skip} is when a voice plays two non-adjacent colours on two consecutive beats.  The direction of a step or skip
is the direction (clockwise or anticlockwise) that a token would travel on the game board
when it moves via the shortest path between the two spaces depicting the two colours involved. A melody must obey the \emph{phrasing} rules: 
\begin{description}
	\item[M1] A repeat must be followed by a step.
	\item[M2] A skip must be followed by a step in the opposite direction.
\end{description}
A \emph{phrase} is a sequence of colours that obey the phrasing rules. Notice that a phrase may not be interrupted by a rest.
\begin{example}
There are two two-beat phrases that start with $R$:
\begin{equation}\nonumber
	[R \quad W] \qquad [R \quad Y]
\end{equation}
and four three-beat phrases that start with $R$:
\begin{equation}\nonumber
	[R \quad B \quad W] \qquad [R \quad G \quad Y] \qquad [R \quad R \quad Y] \qquad [R \quad R \quad W]
\end{equation}
\end{example}

\subsection{Harmony}
\emph{Harmony} is when several different colours are played on the same beat by different voices. A \emph{chord} is a set of colours that obey the \emph{consonance} rule:
\begin{description}
	\item[H1] No more than two colours in a chord may be adjacent.
\end{description}
A three-note chord consists of two adjacent colours and a third isolated colour. This isolated colour is called the \emph{root} of the chord.
\begin{example}
There are five one-colour chords:
\begin{equation}\nonumber
\begin{bmatrix}
R
\end{bmatrix}
\qquad
\begin{bmatrix}
W
\end{bmatrix}
\qquad
\begin{bmatrix}
B
\end{bmatrix}
\qquad
\begin{bmatrix}
G
\end{bmatrix}
\qquad
\begin{bmatrix}
Y
\end{bmatrix}
\end{equation}
ten two-colour chords:
\begin{equation}\nonumber
\begin{bmatrix}
R \\ W
\end{bmatrix}
\quad
\begin{bmatrix}
W \\ B
\end{bmatrix}
\quad
\begin{bmatrix}
B \\ G
\end{bmatrix}
\quad
\begin{bmatrix}
G \\ Y
\end{bmatrix}
\quad
\begin{bmatrix}
Y \\ R
\end{bmatrix}
\quad
\begin{bmatrix}
R \\ B
\end{bmatrix}
\quad
\begin{bmatrix}
W \\ G
\end{bmatrix}
\quad
\begin{bmatrix}
B \\ Y
\end{bmatrix}
\quad
\begin{bmatrix}
R \\ G
\end{bmatrix}
\quad
\begin{bmatrix}
W \\ Y
\end{bmatrix}
\end{equation}
and five three-colour chords:
\begin{equation}\nonumber
\begin{bmatrix}
R \\ G \\ B
\end{bmatrix}
\qquad
\begin{bmatrix}
W \\ Y \\ G
\end{bmatrix}
\qquad
\begin{bmatrix}
B \\ Y \\ R
\end{bmatrix}
\qquad
\begin{bmatrix}
G \\ R \\ W
\end{bmatrix}
\qquad
\begin{bmatrix}
Y \\ W \\ B
\end{bmatrix}
\end{equation}
\end{example}

\subsection{Counterpoint}
\emph{Counterpoint} is when several voices play simultaneously. A pair of voices move in \emph{similar motion} if they both step or skip in the same direction. A pair of voices move in \emph{contrary motion} if they step or skip in opposite directions.  A pair of voices move in \emph{oblique motion} if one voice repeats a colour while the other voice changes colours.  A group of melodies played in counterpoint must satisfy the \emph{voice leading} rules:
\begin{description}
	\item[C1] At least one voice must move on each beat.
	\item[C2] At least on pair of voices must move in contrary or oblique motion on each beat.
\end{description}

\begin{example}
This composition follows the rules established above:
\begin{equation}\nonumber
\begin{bmatrix}
	B & Y & G & W & B \\
	G & G & B & G & Y \\
	R & W & B & W & R
\end{bmatrix}
\end{equation}
\end{example}

\end{document}
