\documentclass{scrartcl}

\usepackage{amsmath, amssymb}

\newtheorem{example}{Example}

\title{SATB}
\subtitle{A Colourful Game of Musical Puzzles}
\author{Michael Purcell}
\date{}

\begin{document}
\maketitle
\section{Rhythm}
Time is divided up into a sequence of \emph{beats}.
\begin{description}
	\item[R1] Each voice can play a single colour on each beat.
	\item[R2] At least one voice must play on every beat.
\end{description}

\section{Melody}
A \emph{melody} consists of a sequence of \emph{prhases}. Each phrase is a sequence of colours played by a single voice that obeys the \emph{phrasing} rules:
\begin{description}
	\item[M1] A repeat must be followed by a step.
	\item[M2] A skip must be followed by a step in the opposite direction.
\end{description}

\begin{example}
There are two two-beat phrases that start with $R$:
\begin{equation}\nonumber
	[R \quad W] \qquad [R \quad Y]
\end{equation}
and four three-beat phrases that start with $R$:
\begin{equation}\nonumber
	[R \quad B \quad W] \qquad [R \quad G \quad Y] \qquad [R \quad R \quad Y] \qquad [R \quad R \quad W]
\end{equation}
\end{example}

\section{Harmony}
\emph{Harmony} is when several different colours are played on the same beat. A \emph{chord} is a set of colours played on the same beat by different voices that obeys the \emph{consonance} rule:
\begin{description}
	\item[H1] No more than two colours in a chord may be adjacent.
\end{description}
\begin{example}
There are five one-colour chords:
\begin{equation}\nonumber
\begin{bmatrix}
R
\end{bmatrix}
\qquad
\begin{bmatrix}
W
\end{bmatrix}
\qquad
\begin{bmatrix}
B
\end{bmatrix}
\qquad
\begin{bmatrix}
G
\end{bmatrix}
\qquad
\begin{bmatrix}
Y
\end{bmatrix}
\end{equation}
ten two-colour chords:
\begin{equation}\nonumber
\begin{bmatrix}
R \\ W
\end{bmatrix}
\quad
\begin{bmatrix}
W \\ B
\end{bmatrix}
\quad
\begin{bmatrix}
B \\ G
\end{bmatrix}
\quad
\begin{bmatrix}
G \\ Y
\end{bmatrix}
\quad
\begin{bmatrix}
Y \\ R
\end{bmatrix}
\quad
\begin{bmatrix}
R \\ B
\end{bmatrix}
\quad
\begin{bmatrix}
W \\ G
\end{bmatrix}
\quad
\begin{bmatrix}
B \\ Y
\end{bmatrix}
\quad
\begin{bmatrix}
R \\ G
\end{bmatrix}
\quad
\begin{bmatrix}
W \\ Y
\end{bmatrix}
\end{equation}
and five three-colour chords:
\begin{equation}\nonumber
\begin{bmatrix}
R \\ G \\ B
\end{bmatrix}
\qquad
\begin{bmatrix}
W \\ Y \\ G
\end{bmatrix}
\qquad
\begin{bmatrix}
B \\ Y \\ R
\end{bmatrix}
\qquad
\begin{bmatrix}
G \\ R \\ W
\end{bmatrix}
\qquad
\begin{bmatrix}
Y \\ W \\ B
\end{bmatrix}
\end{equation}
\end{example}
A three-note chord consists of two adjacent colours and a third isolated colour. This isolated colour is called the \emph{root} of the chord.
\begin{example}
$R$ is the root of the three-note chord $\begin{bmatrix} R \\ G \\ B \end{bmatrix}$
\end{example}

\section{Counterpoint}
\emph{Counterpoint} is when several voices play simultaneously. A pair of voices move in \emph{similar motion} if they both step or skip in the same direction. A pair of voices move in \emph{contrary motion} if they step or skip in opposite directions.  A pair of voices move in \emph{oblique motion} if one voice repeats a colour while the other voice changes colours.  A group of melodies played in counterpoint must satisfy the \emph{voice leading} rules:
\begin{description}
	\item[C1] At least one voice must move on each beat.
	\item[C2] At least on pair of voices must move in contrary or oblique motion on each beat.
\end{description}

\begin{example}
This composition follows the rules established above:
\begin{equation}\nonumber
\begin{bmatrix}
	B & Y & G & W & B \\
	G & G & B & G & Y \\
	R & W & B & W & R
\end{bmatrix}
\end{equation}
\end{example}

\end{document}
